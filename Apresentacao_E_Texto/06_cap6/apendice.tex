\noindent\textbf{Âpendices}
$\!$\\

\section{\textbf{Leitura do Arquivo .RAW}}
\label{sec:CapLTspice}


#!/usr/bin/env python

#-------------------------------------------------------------------------------

#lEITURA DO ARQUIVO .RAW DO LTSPICE
#FOI USADO COMO BASE O CODIGO DO Nuno Brum DE LICENSA FREE, MAS FOI ADICIONADO E ALTERADO ALGUNS METODOS PARA ADAPTAR AO PROBLEMA EM QUESTÃO.
#

#-------------------------------------------------------------------------------




import pandas as pd

from binascii import b2a_hex
from struct import unpack
try:
    from numpy import zeros, array
except ImportError:
    USE_NNUMPY = False
else:
    USE_NNUMPY = True
    print("Found Numpy. WIll be used for storing data")


class DataSet(object):
    """Class for storing Traces."""

    def __init__(self, name, datatype, datalen):
        """Base Class for both Axis and Trace Classes.
        Defines the common operations between both."""
        self.name = name
        self.type = datatype
        if USE_NNUMPY:
            self.data = zeros(datalen)
        else:
            self.data = [None for x in range(datalen)]

    def set_pointA(self, n, value):
        """function to be used on ASCII RAW Files.

        :param n:     the point to set
        :param value: the Value of the point being set."""
        assert isinstance(value, float)
        self.data[n] = value

    def set_pointB(self, n, value):
        """Function that converts a normal trace into float on a Binary storage. This codification uses 4 bytes.
        The codification is done as follows:
               7   6   5   4     3   2   1   0
        Byte3  SGM SGE E6  E5    E4  E3  E2  E1         SGM - Signal of Mantissa: 0 - Positive 1 - Negative
        Byte2  E0  M22 M21 M20   M19 M18 M17 M16        SGE - Signal of Exponent: 0 - Positive 1 - Negative
        Byte1  M15 M14 M13 M12   M11 M10 M9  M8         E[6:0] - Exponent
        Byte0  M7  M6  M5  M4    M3  M2  M1  M0         M[22:0] - Mantissa.

        :param n:     the point to set
        :param value: the Value of the point being set."""

        self.data[n] = unpack("f", value)[0]

    def __str__(self):
        if isinstance(self.data[0], float):
            # data = ["%e" % value for value in self.data]
            return "name:'%s'\ntype:'%s'\nlen:%d\n%s" % (self.name, self.type, len(self.data), str(self.data))
        else:
            data = [b2a_hex(value) for value in self.data]
            return "name:'%s'\ntype:'%s'\nlen:%d\n%s" % (self.name, self.type, len(self.data), str(data))

    def get_point(self, n):
        return self.data[n]

    def get_wave(self):
        return self.data


class Axis(DataSet):
    """This class is used to represent the horizontal axis like on a Transient or DC Sweep Simulation."""

    def __init__(self, name, datatype, datalen):
        super().__init__(name, datatype, datalen)
        self.step_info = None

    def set_pointB(self, n, value):
        """Function that converts the variable 0, normally associated with the plot X axis.
        The codification is done as follows:
               7   6   5   4     3   2   1   0
        Byte7  SGM SGE E9  E8    E7  E6  E5  E4         SGM - Signal of Mantissa: 0 - Positive 1 - Negative
        Byte6  E3  E2  E1  E0    M51 M50 M49 M48        SGE - Signal of Exponent: 0 - Positive 1 - Negative
        Byte5  M47 M46 M45 M44   M43 M42 M41 M40        E[9:0] - Exponent
        Byte4  M39 M38 M37 M36   M35 M34 M33 M32        M[51:0] - Mantissa.
        Byte3  M31 M30 M29 M28   M27 M26 M25 M24
        Byte2  M23 M22 M21 M20   M19 M18 M17 M16
        Byte1  M15 M14 M13 M12   M11 M10 M9  M8
        Byte0  M7  M6  M5  M4    M3  M2  M1  M0
        """
        self.data[n] = unpack("d", value)[0]


    def _set_steps(self, step_info):
        self.step_info = step_info

        self.step_offsets = [None for x in range(len(step_info))]

        # Now going to calculate the point offset for each step
        self.step_offsets[0] = 0
        i = 0
        k = 0
        while i < len(self.data):
            if self.data[i] == self.data[0]:
                #print(k, i, self.data[i], self.data[i+1])
                if self.data[i] == self.data[i+1]:
                    i += 1  # Needs to add one here because the data will be repeated
                self.step_offsets[k] = i
                k += 1
            i += 1

        if k != len(self.step_info):
            raise LTSPiceReadException("The file a different number of steps than expected.\n" +
                                       "Expecting %d got %d" % (len(self.step_offsets), k))

    def step_offset(self, step):
        if self.step_info == None:
            return 0
        else:
            if step >= len(self.step_offsets):
                return len(self.data)
            else:
                return self.step_offsets[step]

    def get_wave(self, step=0):
        return self.data[self.step_offset(step):self.step_offset(step + 1)]


class Trace(DataSet):
    """Class used for storing generic traces that report to a given Axis."""

    def __init__(self, name, datatype, datalen, axis):
        super().__init__(name, datatype, datalen)
        self.axis = axis

    def get_point(self, n, step=0):
        if self.axis is None:
            return super().get_point(n)
        else:
            return self.data[self.axis.step_offset(step) + n]

    def get_wave(self, step=0):
        if self.axis is None:
            return super().get_wave()
        else:
            return self.data[self.axis.step_offset(step):self.axis.step_offset(step + 1)]


class DummyTrace(object):
    """Dummy Trace for bypassing traces while reading"""

    def __init__(self, name, datatype):
        """Base Class for both Axis and Trace Classes.
        Defines the common operations between both."""
        self.name = name
        self.type = datatype

    def set_pointA(self, n, value):
        pass

    def set_pointB(self, n, value):
        pass


class LTSPiceReadException(Exception):
    """Custom class for exception handling"""


class LTSpiceRawRead(object):
    """Class for reading LTSpice wave Files. It can read all types of Files. If stepped data is detected,
    it will also try to read the corresponding LOG file so to retrieve the stepped data.
    """
    header_lines = [
        "Title",
        "Date",
        "Plotname",
        "Flags",
        "No. Variables",
        "No. Points",
        "Offset",
        "Command",
        "Variables",
        "Backannotation"
    ]

    def __init__(self, raw_filename, traces_to_read="*", **kwargs):
        """The arguments for this class are:
    raw_filename   - The file containing the RAW data to be read
    traces_to_read - A string containing the list of traces to be read. If None is provided, only the header is read
                     and all trace data is discarded. If a '*' wildcard is given, all traces are read.
    kwargs         - Keyword parameters that define the options for the loading. Options are:
                        loadmem - If true, the file will only read waveforms to memory
    """
        assert isinstance(raw_filename, str)
        if not traces_to_read is None:
            assert isinstance(traces_to_read, str)

        raw_file = open(raw_filename, "rb")

        # Storing the filename as part of the dictionary
        self.raw_params = { "Filename" : raw_filename } # Initializing the dictionary that contains all raw file info

        startpos = 0  # counter of bytes for

        line = raw_file.readline().decode()

        while line:
            startpos += len(line)

            for tag in self.header_lines:
                if line.startswith(tag):
                    self.raw_params[tag] = line[len(tag) + 1:-1]  # Adding 1 to account with the colon after the tag
                    # print(ftag)
                    break
            else:
                raw_file.close()
                raise LTSPiceReadException(("Error reading Raw File !\n " +
                                            "Unrecognized tag in line %s") % line)

            line = raw_file.readline().decode()
            if line.startswith("Variables"):
                break
        else:
            raw_file.close()
            raise LTSPiceReadException("Error reading Raw File !\n " +
                                       "Unexpected end of file")

        if not ("real" in self.raw_params["Flags"]):
            # Not Supported, an exception will be raised
            raw_file.close()
            raise LTSPiceReadException("The LTSpiceRead class doesn't support non real data")

        self.nPoints = int(self.raw_params["No. Points"], 10)
        self.nVariables = int(self.raw_params["No. Variables"], 10)
        self._traces = []
        self.steps = None
        self.axis = None  # Creating the axis
        # print("Reading Variables")

        for ivar in range(self.nVariables):
            line = raw_file.readline().decode()[:-1]
            # print(line)
            dummy, n, name, var_type = line.split("\t")
            if ivar == 0 and self.nVariables > 1:
                self.axis = Axis(name, var_type, self.nPoints)
                self._traces.append(self.axis)
            elif ((traces_to_read == "*") or
                      (name in traces_to_read) or
                      (ivar == 0)):
                # TODO: Add wildcards to the waveform matching
                self._traces.append(Trace(name, var_type, self.nPoints, self.axis))
            else:
                self._traces.append(DummyTrace(name, var_type))

        if traces_to_read is None or len(self._traces) == 0:
            # The read is stopped here if there is nothing to read.
            raw_file.close()
            return

        self.binary_start = startpos

        # This will make a lazy loading. That means, only the Axis is read. The traces are only read when the user
        # makes a get_trace()
        self.in_memory = False  # point to set it to true at the end of the load

        if kwargs.get("headeronly", False):
            raw_file.close()
            return



        raw_type = raw_file.readline().decode()

        if raw_type.startswith("Binary:"):
            # Will start the reading of binary values
            if "fastaccess" in self.raw_params["Flags"]:
                # A fast access means that the traces are grouped together.
                first_var = True
                for var in self._traces:
                    if first_var:
                        first_var = False
                        for point in range(self.nPoints):
                            value = raw_file.read(8)
                            var.set_pointB(point, value)
                    else:
                        if isinstance(var, DummyTrace):
                            # TODO: replace this by a seek
                            raw_file.read(self.nPoints * 4)
                        else:
                            for point in range(self.nPoints):
                                value = raw_file.read(4)
                            var.set_pointB(point, value)
            else:
                # This is the default save after a simulation where the traces are scattered
                for point in range(self.nPoints):
                    first_var = True
                    for var in self._traces:
                        if first_var:
                            first_var = False
                            value = raw_file.read(8)
                            var.set_pointB(point, value)
                        else:
                            value = raw_file.read(4)
                            var.set_pointB(point, value)

        elif raw_type.startswith("Values:"):
            # Will start the reading of ASCII Values
            for point in range(self.nPoints):
                first_var = True
                for var in self._traces:
                    line = raw_file.readline().decode()
                    # print(line)

                    if first_var:
                        first_var = False
                        spoint = line.split("\t", 1)[0]
                        # print(spoint)
                        if point != int(spoint):
                            print("Error Reading File")
                            break
                        value = float(line[len(spoint):-1])
                    else:
                        value = float(line[:-1])
                    var.set_pointA(point, value)
        else:
            raw_file.close()
            raise LTSPiceReadException("Unsupported RAW File. ""%s""" % raw_type)

        raw_file.close()

        # Setting the properties in the proper format
        self.raw_params["No. Points"] = self.nPoints
        self.raw_params["No. Variables"] = self.nVariables
        self.raw_params["Variables"] = [var.name for var in self._traces]

        # Now Purging Dummy Traces
        i = 0
        while i < len(self._traces):
            if isinstance(self._traces[i], DummyTrace):
                del self._traces[i]
            else:
                i += 1

        # Finally, Check for Step Information
        if "stepped" in self.raw_params["Flags"]:
            self._load_step_information(raw_filename)

    def get_raw_property(self, property_name=None):
        """Get a property. By default it returns everything"""
        if property_name is None:
            return self.raw_params
        elif property_name in self.raw_params.keys():
            return self.raw_params[property_name]
        else:
            return "Invalid property. Use %s" % str(self.raw_params.keys())

    def get_trace_names(self):
        return [trace.name for trace in self._traces]

    def get_trace(self, trace_ref):
        """Retrieves the trace with the name given. """
        if isinstance(trace_ref, str):
            for trace in self._traces:
                if trace_ref == trace.name:
                    # assert isinstance(trace, DataSet)
                    return trace
            return None
        else:
            return self._traces[trace_ref]

    def _load_step_information(self, filename):
        # Find the extension of the file
        if not filename.endswith(".raw"):
            raise LTSPiceReadException("Invalid Filename. The file should end with '.raw'")
        logfile = filename[:-3] + 'log'
        try:
            log = open(logfile, 'r')
        except:
            raise LTSPiceReadException("Step information needs the '.log' file generated by LTSpice")

        for line in log:
            if line.startswith(".step"):
                step_dict = {}
                for tok in line[6:-1].split(' '):
                    key, value = tok.split('=')
                    step_dict[key] = float(value)

                if self.steps is None:
                    self.steps = [step_dict]
                else:
                    self.steps.append(step_dict)
        log.close()
        if not (self.steps is None):
            # Individual access to the Trace Classes, this information is stored in the Axis
            # which is always in position 0
            self._traces[0]._set_steps(self.steps)
            pass

    def __getitem__(self, item):
        """Helper function to access traces by using the [ ] operator."""
        return self.get_trace(item)

    def get_steps(self, **kwargs):
        if self.steps is None:
            return [0]  # returns an single step
        else:
            if len(kwargs) > 0:
                ret_steps = []  # Initializing an empty array
                i = 0
                for step_dict in self.steps:
                    for key in kwargs:
                        ll = step_dict.get(key, None)
                        if ll is None:
                            break
                        elif kwargs[key] != ll:
                            break
                    else:
                        ret_steps.append(i)  # All the step parameters match
                    i += 1
                return ret_steps
            else:
                return range(len(self.steps))  # Returns all the steps


def principal(Circuito):
#if __name__ == "__main__":
    import sys
    import matplotlib.pyplot as plt
    if Circuito =='CTSV mc + 4bitPRBS [FALHA].raw':
        Variavel = 'V(bpo)'
    else:
        Variavel= 'V(vout)'
    if len(sys.argv) > 1:
        raw_filename = sys.argv[1]
    else:
        raw_filename = Circuito
    LTR = LTSpiceRawRead(raw_filename)
    #print(LTR.get_trace_names())
    for trace in LTR.get_trace_names():
        #print(LTR.get_trace(trace))
        Vo = LTR.get_trace(Variavel)
        x = LTR.get_trace(0)  # Zero is always the X axis
        #file = open("Sallen_Vout.txt", "w")
        #print("escrita com sucesso")
        # steps = LTR.get_steps(ana=4.0)
        steps = LTR.get_steps()
        # for step in steps:
       # print("imagem")
        df = {'time': [], Variavel: [],'step': []}
        Dados = []
        time = []
        for step in range(len(steps)):

            ValueVar = Vo.get_wave(step)
            Dados.append(ValueVar)
            #print(ValueVar)
            valueTime = x.get_wave(step)
            time.append(valueTime)



            plt.plot(valueTime, ValueVar, label=LTR.steps[step])

         
        plt.tight_layout(pad=0.5, w_pad=0.5, h_pad=1.0)
        plt.legend()
        plt.title("Dados brutos"+Circuito)
        plt.show()
		
        
        return (LTR, Dados, time)




\section{\textbf{Escrita em um arquivo CSV}}
\label{sec:CapLTspice}

# =-=-=-=-=-=-=-=-
# Projeto de Conclusão de Curso
# Autor: Jéssica Barbosa de Souza
# Descrição : Código para fazer a leitura do arquivo .raw e escrita no arquivo.csv
# =-=-=-=-=-=-=-=-

import pandas as pd
import LTSpice_RawRead as LTSpice
import tslearn
import matplotlib.pyplot as plt
import numpy as np
import visuals as vs
import random


if __name__ == "__main__":
    circuitos = ['CTSV mc + 4bitPRBS [FALHA].raw','Nonlinear Rectfier + 4bit PRBS [FALHA] - 300 - 0.2s.raw','Biquad Highpass Filter mc + 4bitPRBS [FALHA].raw']
    for circuito in circuitos:
        saida,  dados, time = LTSpice.principal(circuito)
        print("leu")
        MaiorIndice = 0
        for dado in dados:
            if len(dado) > MaiorIndice:
                MaiorIndice = len(dado)


        matriz = np.zeros((MaiorIndice, len(dados)))

        i = 0
        j = 0
        for k in range(0, len(saida._traces[10].data)):
            matriz[i][j] = saida._traces[10].data[k]
            if ((saida._traces[10].axis.data[k]) == 0.0) and (k != 0):
                if ((saida._traces[10].axis.data[k - 1]) != 0.0):
                    j += 1
                    i = 0
                else:
                    i += 1
            else:
                i += 1
        file_name = circuito.replace('.raw','.csv')
        dadosOriginais = pd.DataFrame(matriz)
        dadosOriginais.to_csv(file_name, index=False, header=None, sep=';')
        print("escreveu")








