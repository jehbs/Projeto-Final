\begin{center}
\textbf{RESUMO}
\end{center}

%
% O resumo deve ser organizado em apenas um parágrafo mesmo.
% O número de folha é o número de páginas do PDF -2. Isto ocorre pois na versão final (capa dura) a capa é removida e as duas primeiras páginas são impressas em uma % folha apenas (frente e verso).
%

$\!$\\

\hspace{-1.3cm}\textbf{Souza}, Jéssica B \textit{Detecção de falhas em circuitos analógicos utilizando aprendizado de máquinas}. 114 f. Projeto de Graduação~(Graduação em Engenharia elétrica com ênfase em sistemas eletrônicos) - Faculdade de Engenharia, Universidade do Estado do Rio de Janeiro~(UERJ), Rio de Janeiro, 2018.

\vspace{.2cm}

Este trabalho apresenta uma aplicação real de sistemas inteligentes em projetos de eletrônica. Foram utilizadas técnicas de aprendizado supervisionado para classificar falhas em circuitos analógicos simulados no LTSpice. Entre as duas pontas existe um complexo processo de Extração, Limpeza e pré processamento desses dados. Para isso foram usadas técnicas de \textit{data Cleaning}, \textit{data Mining} e aplicou-se consagrados algoritmos como o PAA (Aproximação Agregada por Partes) e o PCA (Análise das Componentes Principais) para a redução. Ao fim, foram aplicados 10(dez) diferentes algoritmos de aprendizado para a classificação dessas falhas em busca do mais eficiente e robusto para o sistema, sendo todos submetidos a intensos testes também apresentados neste trabalho. Todo o código foi desenvolvido em Python.

\vspace{1cm}

\hspace{-1.3cm}Palavras-chave: Aprendizado de máquina, LTspice, Classificação, Python.