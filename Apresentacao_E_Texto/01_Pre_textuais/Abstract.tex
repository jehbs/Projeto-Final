\begin{center}
\textbf{ABSTRACT}
\end{center}

$\!$\\

% O resumo em inglês deve ser organizado em apenas um parágrafo mesmo.


\hspace {-1.3cm} \textbf {Souza}, Jessica B \textit{Detection of faults in analog circuits using machine learning}. 114 f. Graduation Project ~ (Graduation in Electrical Engineering with emphasis on electronic systems)  - Faculdade de Engenharia, Universidade do Estado do Rio de Janeiro~(UERJ), Rio de Janeiro, 2018.

This paper presents a real application of intelligent systems in electronics projects. Supervised learning techniques were used to classify faults in simulated analog circuits in LTSpice. Between the two ends there is a complex process of Extraction, Cleaning and pre-processing of this data. For this purpose, data cleaning techniques were used, data mining was used, and algorithms such as the PAA (Approximation Aggregated by Parts) and PCA (Major Component Analysis) were applied for the reduction. Finally, ten (10) different learning algorithms were applied to the classification of these faults in search of the most efficient and robust for the system, all of which were submitted to intense tests also presented in this work. All code was developed in Python.

\vspace{1cm}

\hspace{-1.3cm}Keywords: Machine-Learning, classification , LTSpice, Python.