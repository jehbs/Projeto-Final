\addtocounter{page}{+1}
\begin{center}

Jéssica Barbosa de Souza

\vspace{1cm}

\textbf{Detecção de Falhas em Circuitos Analógicos utilizando métodos de aprendizado de máquinas}

\end{center}

\vspace{.6cm}

\begin{flushright}
\parbox{8cm}{
\singlespacing{Projeto de graduação apresentado, como requisito parcial para obtenção do grau de Engenheiro Eletricista, à Faculdade de Engenharia, da Universidade do Estado do Rio de Janeiro.}
}
\end{flushright}

\vspace{.9cm}


% insira abaixo a data de sua defesa
% Caso não tenha defendido ainda, deixe em branco

\noindent Aprovado em 08 de Outubro de 2018

\noindent Banca Examinadora:


%
%
% Os professores da UERJ DEVEM ser citados primeiro, independente de quem seja o orientador.
%
%



\vspace{1.0cm}

\begin{flushright}
\parbox{13cm}{

\singlespacing

\hrulefill \\

\vspace{-.3cm}
Prof. Dr. Jorge Luís Machado do Amaral (Orientador)
\newline
Faculdade de Engenharia da UERJ
\vspace{1.9cm}

\hrulefill \\

\vspace{-.3cm}
Prof. Dr. Marco Aurélio Botelho da Silva  
\newline
Faculdade de Engenharia da UERJ
\vspace{1.9cm}

\hrulefill \\

\vspace{-.3cm}
Prof. Dr. Douglas Mota Dias
\newline
Faculdade de Engenharia da UERJ
\vspace{.6cm}





}
\end{flushright}
\vfill

\begin{center}
Rio de Janeiro\linebreak 2018
\end{center}
